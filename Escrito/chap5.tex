%!TEX root = memoria.tex
\chapter{Capítulo 5}

\section{Validación y Análisis de los Resultados}

\textit{\textbf{Recomendaciones generales:}}
\begin{itemize}
  \item \textit{Presenta los resultados interpretando los datos a la luz del marco teórico planteado, dependiendo de la pregunta de investigación y los objetivos de la misma.}
  \item \textit{Recuerda que los datos no son los resultados. Los resultados vienen del análisis de los datos para encontrarles sentido. Se apoyan en los datos, pero son interpretaciones del estudiante que requieren de argumentación para que tengan sentido. Además, se debe presentar la respuesta que el estudiante da a la pregunta de investigación, con la argumentación que lo lleva a dar esta respuesta.}
\end{itemize}

\section{Conclusiones}

\textit{La función principal de las conclusiones es exponer los hallazgos, observaciones y posibles retos que se desprendan de la investigación realizada. Esta sección nunca puede, como parte formal, ser más extensa que la investigación misma. Una buena sección de conclusiones expondrá de manera puntual y ordenada una síntesis de la investigación realizada a partir de los resultados obtenidos en relación con la hipótesis que dio inicio al trabajo. Al exponer los resultados, debe poder apreciarse si se trata de resultados esperados, si difieren de la expectativas, si son parciales o insuficientes. Es decir, los resultados deben referenciarse con el punto de partida de la investigación. Esto hace que, entre la sección de introducción y la de las conclusiones, haya un vínculo evidente.}\\

\textit{Es importante considerar que las conclusiones no constituyen un resumen de los capítulos o partes de la tésis. En esta sección tampoco es apropiado introducir nuevos elementos teóricos ni hay lugar para retomar el objeto de estudio desde otras perspectivas.}\\

\textit{La parte más valiosa de las conclusiones es la reflexión que se deriva de los resultados obtenidos. No es suficiente listar los resultados. A partir de ellos hay qye elaborar observaciones y, sobre todo, nuevas preguntas para futuras investigaciones que permitan aportar en la contrucción del conocimiento.}