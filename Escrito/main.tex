%%%%%%%%%%%%%%%%%%%%%%%%%%%%%%%%%%%%%%%%%%%%%%%%%%%%%%%%%%%%%%%%%%%%%%%%%%%%%%
%	Plantilla para Memorias y Tesis, UTFSM, Chile
%	=============================================
%
%	AUTOR:
%		Jaime C. Rubin-de-Celis <jaime@rubin-de-celis.com>
%
%	FECHA:	
%   $Date: 2015-11-28 19:05:22 -0400 (Sat, 28 Nov 2015) $
%
% VERSIÓN:
%   2.1
%
%	LICENCIA:
% 	Copyright (c) 2012-2015, Jaime C. Rubin-de-Celis
%   Licensed under the Educational Community License version 1.0 (SE ADJUNTA)
%   NOTA: Las imágenes son propiedad intelectual de la UTFSM.
%
% USO:
%   Ver archivo adjunto README.md.
%   Si todo lo demás falla, recuerde "RFM".
%
%%%%%%%%%%%%%%%%%%%%%%%%%%%%%%%%%%%%%%%%%%%%%%%%%%%%%%%%%%%%%%%%%%%%%%%%%%%%%%


%---------------------------------------------------------------------------
%%% DOCUMENT CLASS
\documentclass[
	12pt,
	letterpaper,
	oneside]{book}
%---------------------------------------------------------------------------


%---------------------------------------------------------------------------
%%% LANGUAGE: Spanish
\usepackage[spanish,spanish]{babel}
%---------------------------------------------------------------------------


%---------------------------------------------------------------------------
%%% STANDARD USM THESIS STYLE
% Este archivo contiene la configuración básica de la plantilla.
\usepackage{memoriaUSM}
\usepackage{tracklang}
\usepackage{glossaries}
\usepackage{booktabs}
\makeglossaries

% Control del nivel de profundidad de la Tabla de Contenidos (TDC).
\setcounter{tocdepth}{3}       % {3} incluirá subsubseccions en la TDC. 
% Control de la numeración de secciones, subsecciones y subsubsecciones
\setcounter{secnumdepth}{3}    % {3} numerará las subsubsecciones.
%---------------------------------------------------------------------------

%---------------------------------------------------------------------------
%%% ENCODING
% Este documento está escrito usando caracteres Unicode (UTF8) (MacOSX)
% Por lo que la siguiente línea es necesaria para reconocer los acentos
% y otros caracteres en español.
%	Si ve caracteres extraños en el PDF (en Windows o MAC) pruebe 
% eliminando/comentando las siguientes líneas:
\usepackage[utf8x]{inputenc}

%%% MICROTYPE: Subliminal refinements
\usepackage{microtype}
% ATENCIÓN: Algunos usuarios de Windows(!) pueden presentar problemas para
% compilar su documento usando 'microtype'. Comentar o eliminar esta línea
% SOLO si experimenta problemas. Su uso para mejorar la diagramación 
% automática es recomendable.
%---------------------------------------------------------------------------


%---------------------------------------------------------------------------
%%% <<<<<<<<<< CONFIGURACIÓN >>>>>>>>>>>>>
%	Cambie los siguientes campos según sea necesario:

\newcommand{\TheTitleOne}        {ANÁLISIS Y EVALUACIÓN}
\newcommand{\TheTitleTwo}        {DE LA TRANSFERENCIA DE MENSAJES}
\newcommand{\TheTitleThree}      {EN ARQUITECTURAS DE MICROSERVICIOS}
\newcommand{\TheAuthor}          {BRIAN ALEJANDRO VIDAL CASTILLO}
\newcommand{\TheGrade}           {INGENIERÍA DE EJECUCIÓN EN INFORMÁTICA} % Elegir "O" ó "A"
\newcommand{\TheAdvisor}         {RAÚL MONGE ANDWANTER}
\newcommand{\TheCoAdvisor}       {HERNÁN ASTUDILLO ROJAS}
%\newcommand{\TheScndCoAdvisor}   {SRTA. XXXXXXX XXXXXXXX X.} % Opcional
\newcommand{\TheCity}            {VALPARAÍSO}
\newcommand{\TheDate}            {NOVIEMBRE 2016}

\newcommand{\chapquote}[3]{\begin{quotation} \textit{#1} \end{quotation} \begin{flushright} - #2, \textit{#3}\end{flushright} }

%%% >>>>>>>>>>>>> CONFIGURACIÓN <<<<<<<<<< 
%---------------------------------------------------------------------------


%---------------------------------------------------------------------------
%%% PDF OPTIONS (HYPERREFERENCES)
\RequirePackage[%
		colorlinks=true,
		linkcolor=blue,
		urlcolor=blue,
		citecolor=blue,
		breaklinks=true,
		backref=section]{hyperref}	% for colored links (urls and table of contents)
\hypersetup{  
  pdfinfo={  
    Subject={Memoria Departamento de Informática, UTFSM},
    Keywords={Memoria} {Departamento de Informática} {UTFSM},
		Producer={JCR's Template, http://www.rubin-de-celis.com/},
		Licence={http://www.rubin-de-celis.com/LICENSE}
		pdfpagemode=FullScreen,
		pdfmenubar=false,
		pdftoolbar=false
  }  
}
% IMPORTANT: Do not edit next lines!
\hypersetup{  
  pdfinfo={  
    Title={\TheTitleOne\ \TheTitleTwo\ \TheTitleThree},  
    Author={\TheAuthor}
  }  
}
%---------------------------------------------------------------------------


%---------------------------------------------------------------------------
%%% CHAPTER HEADINGS
%
% Alternative 1: (Requires fontenc!)
\usepackage{titlesec}
\titleformat{\chapter}[hang]{\Huge\bfseries}{\thechapter\hsp\textcolor{gray80}{|}\hsp}{0pt}{\Huge\bfseries}
%
% Alternative 2:
% Options: Sonny, Lenny, Glenn, Conny, Rejne, Bjarne, Bjornstrup
%\usepackage[Conny]{fncychap}
%---------------------------------------------------------------------------

%%%%%%
\makeatletter		% Do not move!
%---------------------------------------------------------------------------
%%% WATERMARK
% \usepackage{graphicx} % required, if not loaded already!
\usepackage{xcolor}
\usepackage{type1cm}
\usepackage{eso-pic}
\AddToShipoutPicture{%
            \setlength{\@tempdimb}{.34\paperwidth}%
            \setlength{\@tempdimc}{.6\paperheight}%
            \setlength{\unitlength}{1pt}%
            \put(\strip@pt\@tempdimb,\strip@pt\@tempdimc){%
            %\makebox(0,0){\rotatebox{45}{\textcolor[gray]{0.9}%
            %{\fontsize{6cm}{6cm}\selectfont{DRAFT}}}}%
            \includegraphics{figures/logousm_watermark.jpg}
            }%
}
%---------------------------------------------------------------------------

%\makeatother

%---------------------------------------------------------------------------
%%% TYPESETTING AIDS
% this can be deleted for a final document.
%\usepackage{lipsum}			  % automated text generation
%\usepackage{blindtext}	  % automated text generation
%\usepackage{showframe} 	% Show frames (formatting aid)
%---------------------------------------------------------------------------



%%%%%%%%%%%%%%%%%%%%%%%%%%%%%%%%%%%%%
%	DOCUMENT
%%%%%%%%%%%%%%%%%%%%%%%%%%%%%%%%%%%%%
\begin{document}


\pagestyle{plain}				% Reset Header, Footer, Page Number

%---------------------------------------------------------------------------
%%% PORTADA
\thispagestyle{empty}		% Hide Header, Footer, Page Number
\begin{center}
\begin{spacing}{1}
{\large UNIVERSIDAD TÉCNICA FEDERICO SANTA MARÍA}\\
DEPARTAMENTO DE INFORMÁTICA\\
VALPARAÍSO - CHILE
\end{spacing}

\vspace{12mm}


\includegraphics[height=60mm]{figures/logousm.png}


\vspace{15mm}
\begin{spacing}{1.5} 
\textbf{\large\TheTitleOne}\\
\textbf{\large\TheTitleTwo}\\
\textbf{\large\TheTitleThree}
\end{spacing}

\vspace{20mm}
\textbf{\large\TheAuthor}

\vspace{12mm}
\begin{spacing}{1.25} 
MEMORIA PARA OPTAR AL TÍTULO DE\\
\TheGrade
\end{spacing}

\vspace{15mm}
\begin{table}[h]
  \begin{center}
  \begin{tabular}{ l c l }
  PROFESOR GUÍA & : & \TheAdvisor \\
  PROFESOR CORREFERENTE & : & \TheCoAdvisor \\
  \ifdefined\TheScndCoAdvisor
  PROFESOR CORREFERENTE & : & \TheScndCoAdvisor{}
  \fi
  \end{tabular}
  \end{center}
\end{table}
\vfill
\large\TheDate
\end{center}
					% Archivo portada.tex
\newpage
%---------------------------------------------------------------------------


\frontmatter            % Compose the FrontPage!


%---------------------------------------------------------------------------
%%% AGRADECMIENTOS
\thispagestyle{empty} 		% Hide Header, Footer, Page Number
\section*{Agradecimientos}
%!TEX root = memoria.tex

En esta sección escribiré los agradecimientos.
	% Archivo agradecimientos.tex
\newpage
%---------------------------------------------------------------------------


%---------------------------------------------------------------------------
%%% DEDICATORIA
\thispagestyle{empty}		% Hide Header, Footer, Page Number
\vspace*{4cm}
\hfill
\begin{minipage}[t]{0.45\linewidth}
\emph{%!TEX root = memoria.tex

Dedicatoria \\
En esta sección se incluirá la dedicatoria.
}	% Archivo dedicatoria.tex
\end{minipage}
\newpage
%---------------------------------------------------------------------------


%---------------------------------------------------------------------------
%%% RESUMEN EJECUTIVO
\section*{RESUMEN EJECUTIVO}
%!TEX root = memoria.tex

% introducir arquitecturas de microservicios, promesas y problemas
Las Arquitecturas de Microservicios proponen la separación de grandes sistemas en pequeños componentes de responsabilidad reducida que poseen su propio almacén de datos, una interfaz de comunicación bien definida y son desarrollados por equipos pequeños e independientes. Un sistema de mediana embergadura suele poseer decenas de microservicios, los que requieren un entorno computacional adaptable a la carga del sistema y estrategias de comunicación entre ellos.
% presentar conclusiones arquitectónicas
Cada mensaje que viaja de un microservicio a otro debe ser serializado en el origen, transportado a través de la red, ruteado y deserializado en destino. La elección de tecnologías y estrategias para realizar dichas operaciones es vital para el perfecto desempeño del sistema y de los programadores, cuyos criterios dependen de la naturaleza del problema. Sin embargo, la estabilización del mercado de herramientas permite generar una matriz que ayude a los arquitectos de software a escoger las herramientas que han probado ser aptas en la industria y alejarse de aquellas combinaciones que han sido insuficientes o sobre complejizadas.
% presentar conclusiones de desarrollo de sistemas
La correcta elección puede acelerar el desarrollo de software, optimizar recursos humanos y mejorar la comunicación entre equipos, además de permitir la introducción de nuevos programadores de manera controlada, aislada y fomentar la innovación dentro de las empresas.

\paragraph{Palabras Claves:} 
Arquitecturas de Microservicios, serialización binaria, protocolos binarios, desarrollo de software basado en mensajes

\vspace{20mm}
					% Archivo resumen.tex
\newpage
%---------------------------------------------------------------------------


%---------------------------------------------------------------------------
%%% ABSTRACT
\section*{ABSTRACT}
%!TEX root = memoria.tex

% introducir arquitecturas de microservicios, promesas y problemas
Micro-services Arquitectures advocates the separation (shift from monoliths to smaller units...) of big systems to tiny components with single responsability, isolated databases, a single comunication interface; usually developed by small and independent teams. A medium-sized system usually consists of dozens microservices, which require a computationally adaptable environment and comunication strategies between them.
% presentar conclusiones arquitectónicas
Each message that travels from one microservice to another must be serialized on origin, transported trough the network, routed and deserialized on destination. The choice of technologies ans strategies to fullfill those operations is vital to the optimal performance of the system and the programmers. The criteria to choose depends of the nature of the solution. Nevertheless the stabilization of tools allows the construction of an arquitectural matrix that can be used by software architects to choose the tools that have been proven on the industry and stay away from the combinations that have been insuficient or overly complex.
% presentar conclusiones de desarrollo de sistemas
The right choice can accelerate software development, optimize human resources and improve comunication between teams, it also allows the introduction of new programmers in a controlled and isolated way, fostering innovation inside the organization.

\paragraph{Keywords.}
Micro-services Arquitectures, binary serialization, binary protocols, message-diven development

\newpage
% Archivo abstract.tex
%---------------------------------------------------------------------------


%---------------------------------------------------------------------------
%%% GLOSARIO
\section*{GLOSARIO}
%!TEX root = memoria.tex

\newglossaryentry{entry}
{
  name={entry},
  description={my entry description.}
}

\printglossaries

Canal de mensajes: Es el medio por el cual un productor de mensajes y un consumidor se comunican

Mensaje: Unidad básica en la construcción de arquitecturas basasdas en mensajes? microservicios? No son necesarios los microservicios, pero sí el paso de mensajes. Microservicios sólo dramatizan esta situación. Ahí la utilidad.
Un mensaje es una estructura de datos definida, sin comportamiento propio que es intercambiado entre componentes de un sistema.

Microservicio: Componente de un sistema. Posee una única responsabilidad y almacén propio de datos. Toda comunicación es a través de una interfaz bien definida.

Arquitectura de microservicios: Patrón de arquitectura de software donde los sistemas son construidos por numerosos componentes responsables de una única tarea en lugar de un gran monolito con múltiples responsabilidades.

Programación asíncrona: Capacidad de un sistema de entregar un contendor para un dato que eventualmente estará disponible

Reactividad: Capacidad de un sistema para responder ante la eventual disponibilidad de un dato

Arquitecturas basadas en mensajes: Patrones de arquitectura de software donde la comunicación es a través del paso de estructuras de datos conocidas tanto por el origen y el destino.

Serialización: Proceso reversible por el cual una estructura de datos es transformada a una representación transportable por la red

Deserialización: Proceso por el cual una estructura de datos es trabsformada desde su representación a su estado original

Protocolos de transporte: Conjunto de reglas por las cuales una estructura de datos es negociada, entregada o rechazada (ELABORAR) 

The difference being that messages are directed, events are not — a message has a clear addressable recipient while an event just happen for others (0-N) to observe it.

Message oriented applications by default come with removal of shared memory. Publisher and subscriber don't need to share a memory space. On the other hand, all the other features (i.e. order, method name coupling and such) are not necessities.					% Archivo glosario.tex
%---------------------------------------------------------------------------


%---------------------------------------------------------------------------
%%% TABLA DE CONTENIDOS / FIGURAS / CUADROS
\begin{spacing}{1}
\tableofcontents
\listoftables
\listoffigures
\end{spacing}
\newpage
%---------------------------------------------------------------------------


%---------------------------------------------------------------------------
%%% INTRODUCCIÓN
\section*{INTRODUCCIÓN}
%!TEX root = memoria.tex

\subsection{Contexto:Motivo}
{\it Justo después del resumen inicia la introducción, en la cual el tesista ubica al lector en el contexto mundial, nacional, así como señala la importancia del tema que se ha seleccionado en el área de su disciplina y así mismo guía al lector hacia la estructura de la tesis narrando lo que encontrará en casa uno de los capítulos del documento.}

La rama de la Informática que se dedica al diseño e implementación de sistemas distribuidos ha tenido una década sin tregua.

Desde la masificación de los navegadores web y los dispositivos móviles, se ha trabajado incesablemente por entregar mejores herramientas a quienes implementan servicios complejos, distribuidos en distintas regiones geográficas; sistemas que deben mantenerse en coordinación y coherencia. La práctica más común, hace un par de años, era desarrollar enormes sistemas con grandes almacenes de datos y alojarlos en costosa infraestructura.

Latinoamérica no puede darse esos lujos [1].

La revolución cultural informática que ha traído el uso de contenedores se ha reflejado en la forma de construir aplicaciones distribuidas; hoy ya no diseñamos monolitos y hemos adoptamos la cultura de construir un gran número de servicios informáticos con tareas muy específicas que, en conjunto, entregan un gran nivel de resilencia, elasticidad y por sobre todo, claridad mental a quienes los desarrollan.

Sin embargo, tal cantidad de servicios, cuyo número siempre crece (tal como nos indica la experiencia) se ven enfrentados a una disyuntiva; por un lado requieren de un fuerte contrato acerca de sus responsabilidades y las tareas que son capaces de recibir (nuestros mensajes) y por otro lado demandan un entorno que les permita iterar rápidamente sin acoplarlos demasiado al sistema como un todo.

Muchas de las necesidades de estos pequeños pero numerosos servicios han sido satisfechas por gigantes tecnológicos que saben de primera fuente lo complejo que resulta orquestarlos, distribuirlos, mantenerlos operativos, permitirles fallar elegantemente y más importante aún: mantener una tasa creciente de innovación. Sin embargo, muy poco se ha estudiado acerca de la forma y la lengua en que se comunican.

La principal barrera en la implementación de soluciones ha sido hasta ahora la enorme diversidad de herramientas, los lenguajes de programación y especificidad en cada solución. Este trabajo busca transformar esta debilidad en una oportunidad para los arquitectos de software [2].

Hoy en día el método más utilizado en la comunicación inter-sistemas es a través de texto plano, a nivel de transporte vía HTTP o a través de protocolos binarios que transportan estructuras codificadas en… texto plano, lo que ha dado a luz a diversas técnicas de programación defensiva y, sin ir más lejos, pérdida de productividad en los desarrolladores que deben preocuparse por la “forma” que traen los datos que necesitan en lugar de simplemente utilizarlos [3].

Y es allí donde radica la oportunidad de abrazar la diversidad de lenguajes de programación y protocolos de transporte, utilizando datos estructurados o semi estructurados en un dialecto neutral (a los lenguajes de programación).

A pesar de no ser el tema de estudio en esta memoria, las diferencias en rendimiento computacional (considerando cada tecnología por separado) son abismantes [4] [4.1].

Entonces, ¿Por qué seguimos comunicando sistemas a través de texto? ¿No existen acaso convenciones que nos ayuden a diseñar tales sistemas? Y más importante aún ¿Por qué estamos usando el camino más lento y endeble para construir sistemas igual de endebles?

En el desarrollo de esta memoria propongo desarrollar argumentos que motiven a los arquitectos de software a utilizar protocolos binarios para transportar estructuras de datos bien definidas, versionables, además de utilizar herramientas de generación de código para automatizar estas tareas, así como también formas de integrar estos procesos en la cadena de producción de software, para potenciar la integración continua.

A la vez, esta memoria pretende ser la antesala al fenómeno del Internet de las Cosas en Latinoamérica, cuyo núcleo es la comunicación entre pequeños sistemas [5]. % Archivo introduccion.tex
%---------------------------------------------------------------------------


%%%%%%%%%%%%%%%%%%%%%%%%%%%%%%%%%%%%%
%	Cuerpo Principal (Main Matter)
%%%%%%%%%%%%%%%%%%%%%%%%%%%%%%%%%%%%%
\mainmatter
\pagestyle{fancy}

%!TEX root = memoria.tex
\chapter{Capítulo 1}

FOCO: Estudiar cómo se comportan los protocolos de transferencia actuales en el escalado de microservicios cuando se agregan servicios nuevos y, a la vez, aumentan las peticiones.

http://blog.wix.engineering/2015/07/14/building-a-scalable-and-resilient-architecture/

\section{Definición del Problema}

La infraestructura necesaria para el funcionmiento de internet es demasiado costosa en términos económicos, ambientales, climáticos y políticos.

Los sistemas que se mantienen relevantes se ven enfrentados a complejidad en la modularización de sus componentes, falta de resilencia, ineficiencias en recursos humanos y un alto costo de planificación y desarrollo.

A lo largo de los años, diversas patrones de programación y arquitecturas de software han demostrado lo complejo que es crear sistemas distribuidos confiables. Aún más, distribuir un sistema que inicialmente fue creado como una única entidad genera nuevas necesidades del sistema que antes no existían (COMPLEJIDAD ASOCIADA).

Los protocolos de internet están basados principalmente en TCP, a pesar de existir mecanismos más ágiles y bien adoptados como UDP o exóticos como OSI \colorbox{green}{REFERENCIA}. Una de las razones para este panorama es la comprobación de entrega de paquetes de datos que ofrece TCP y la posibilidad de utilizar tecnologías fundacionales existentes para crear tecnologías más sofisticadas. Luego, es deseable generar arquitecturas de software cuyo principal mecanismo de transporte sea TCP.

Es deseable también un protocolo que esté mejor destinado a un patrón como REST (Representational State Transfer, Transferencia de Estado Representacional) que HTTP como sucede en la actualidad, donde los clientes de un sistema no poseen acceso a los eventos que notifiquen los cambios de estado en estas entidades. Es rescatable el amplio uso del patrón, lo que ha ha entregado a la comunidad un conjunto mínimo de normas sobre las cuales actuar ante entidades de datos, resultando de librerías informáticas de gran calidad, gran participación de individuos en proyectos de código abierto y curvas de aprendizaje menos sinuosas y menos elevadas.

Éticamente, es correcto utilizar protocolos que sean entendibles por humanos, de tal forma que la información en tránsito pueda ser auditada por la ciudadanía, en caso de ser necesario. Sin embargo, el flujo de información, una vez ésta ingresa en un sistema, está libre de tales restricciones, dando cabida al uso holgado de transportes y codificaciones binarios, cuando suponen una ventaja competitiva o mejora en rendimiento.

Y si consideramos recursos humanos, hay varios aspectos relevantes que quedan al descubierto; por un lado, las posibilidades de hardware empujan a los los lenguajes y librerías informáticas a introducir elementos primitivos que permitan la programación paralela, para utilizar los núcleos disponibles, al mismo tiempo que permitan la construcción de sistemas distribuidos. Por otro lado, están los equipos de desarrollo, que históricamente han sido organismos en crecimiento constante, encargados cada vez de sistemas más difíciles de manejar. Todo esto conlleva complejidad y altos costos.

\textit {Relevancia del problema}\\

Consolidar las arquitecturas de microservicios y crear un par de combinaciones eficientes, sólidas y SIMPLES de transporte y serialización.

\textit {Por qué este problema es relevante para su contexto local e institucional?}\\

Evaluar contemporáneamente un paisaje tan cambiante en esta década, como lo es la programación distribuida, es muy relevante cuando Internet es sólo la antesala de el Internet de las cosas, que ya posee implementaciones para nichos muy particulares (como el Internet de los Automóviles) o movimientos altamente políticos y sociales, como son las redes en malla de libre acceso (como el caso de Hyperbórea).

\begin{itemize}
  \item Requiermientos de arquitecturas modernas
  
  \begin{itemize}
    \item Larga vida del software
    \item Priman atributos de calidad (no-funcionales)
    \item Retrasar diseño (hasta que los problemas existan)
    \item Cambios!
    \item Modularidad para automatización (build, test, deploy)
    \item Reflejar la estructura organizacional
  \end{itemize}

  \item ¿Qué arquitecturas se ven afectadas?
  \item ¿Cuál es la influencia del panorama actual de Internet? (navegadores, HTTP, APIs)
  \item ¿Cuáles son los desafíos que trae IoT en este campo? (tostadores con conectividáh)
  \item ¿Qué otros transportes han sido efectivos en solucionar el problema?
  \item ¿Cuáles son los puntos más dolorosos para la industria hoy?
  \item ¿Cuáles serán mañana?
\end{itemize}


\section{Objetivos}

\subsection{Objetivo General:}
Evaluar el comportamiento de los principales protocolos de transferencia de mensajes \colorbox{green}{Ser consistente con el nombre del tema} cuando una arquitectura de microservicios se ve enfrentada a la necesidad de escalar sus operaciones tanto en inclusión de nuevos servicios tanto como un mayor afluente de peticiones.
El resultado principal es la elaboración de un conjunto de estrategias para arquitectos de software donde se relacione el dominio del problema, su arquitectura subyacente, aplicaciones y tecnologías óptimas para llevar a cabo la solución.

\subsection{Objetivos Específicos:}
Concretamente, en el marco de la creación de mensajes que deben ser transportados, comprendidos y ejecutados por varios sistemas, el trabajo espera:

\begin{itemize}
  \item Generar una retrospectiva histórica de protocolos usados en la industria
  \item Estudiar y aplicar técnicas de serialización neutral en diversos lenguajes de programación cooperantes
  \item Establecer estrategias para la elección de transporte y serialización
  \item Crear una matriz arquitectural de herramientas neutrales a los lenguajes de programación
  \item Establecer criterios para la elección de protocolos textuales o binarios
  \item Contrastar la serialización binaria y la basada en texto plano
  \item Descubrir potenciales beneficios en la optimización de recursos humanos
\end{itemize}

\section{Retrospectiva Histórica}

\subsection{Patrones}

\begin{itemize}
  \item RPC %como abstracción al paso real de mensajes, incluyendo serialización y deserialización así como contrato compartido (client stub) A remote invocation mechanism alone, however, is not suficient for building distributed programs. Objects that should reside on separate nodes somehow need to get to these nodes in the first place, and, having been placed onto the desired nodes, the objects need to make initial contact with each other %
  \item CORBA (OMG 2000) %CORBA is a pragmatic approach to providing a distribution infrastructure for enterprise applications, where the main focus is on interoperability between heterogeneous platforms and programming languages. As such, distribution transparency is only a minor objective of CORBA. Similar as with Java/RMI, pretty much the only thing that is indeed distribution-transparent is the actual invocation of a remote object. In practically all other areas, distribution-related issues are completely visible in the source code. A CORBA application must necessarily be completely different from a centralized program that performs the same task.%
\end{itemize}

\subsection{Arquitecturas}

\begin{itemize}
  \item SOA \colorbox{green}{PATRÓN}
  \item Monolitos
  \item Microservicios
  \item Sistemas auto-contenidos
\end{itemize}

\colorbox{green}{acerca de microservicios}
% Seriously though - for many businesses, the biggest cost for software isn't the software anymore. It's the bandwidth, hardware, CDN costs, etc. Now that everyone has a mobile device, there's just that much more traffic. And that will only get worse as your toaster gets its own internet connectivity.

% So businesses are looking to manage those costs. Specifically, they're trying to handle the business problem of \"if this thing blows up, how can I serve millions of people getting/using my software - without paying ahead of time for the servers to serve millions of people getting/using my software?\". 

% While HTTP and REST are preferred for synchronous communication, it’s becoming increasingly popular to use asynchronous communication between microservices. Many consider the Advanced Message Queuing Protocol (AMQP) standard as the preferred protocol, in this regard. Developing microservices with an asynchronous communication model, while sometimes a little more complex, can have great advantages in terms of minimizing latency and enabling event-driven interactions with applications.

% In the market today, RabbitMQ and Apache Kafka are both commonly used message bus technologies for asynchronous communication between microservices. Also, if the message-passing is done on the same host, then the containers can communicate with each other by way of system calls, as they all share the same kernel.

% While system and application requirements continue to evolve, the methodology behind how we solve these problems is often based on older models and patterns. As mentioned before, microservices architecture has its roots in models like COM, COBRA, EJB and SOA, but there are still some rules to live by when creating microservices utilizing current technologies. While the ten best practices we’ve laid out here are not entirely comprehensive, they are core strategies for creating, migrating and managing microservices.

% On the other hand, with implicit systems (where distribution is handled in the run-time system), transparency is simply not an issue. By their nature, distribution is completely invisible to the programmer, whic h mak es them app ealingly elegan t. The big problem of these platforms is their lack of e±ciency , whic h researc hers are trying to comp ensate for by ever higher degrees of clev er automatic optimizations. Despite this, implicit platforms have not really been put to use outside of academia yet

% Waldo et al. argue that distributed and non-distributed programming cannot be uniØed, and they do so with considerable rhetoric effort.

% A Note on Distributed Computing

% The di±cult part, Waldo et al. con tinue, lies in four distinct areas where the local and the distributed case are separated by insurmoun table diÆerences. ≤ Latency. A remote metho d invocation tak es between four and Øve orders of magnitude longer than a local metho d invocation, and the curren t trends in both pro cessor speed and net work latency suggest that this will not fundamen tally change in the future. As a consequence, Waldo et al. argue, not paying atten tion to distribution from the earliest phases of dev elopmen t may lead to designs with insurmoun table performance problems. It must be decided righ t from the beginning what objects can be made remote and what objects must be clustered together" (op.cit., page 5). ≤ Memory access. Direct memory addresses are not valid outside a single address space. Waldo et al. conclude that if local and distributed computing are uniØed, this means that programmers must not use address-space-relativ e pointers. However, this restriction could only be enfor ced if the abilit y to get address-space-relativ e pointers were completely remo ved from the programming language. This, on the other hand, would require pro- grammers to learn a new style of programming, and thus give up the complete transparency between local and distributed computing. ≤ Partial failur e. In a distributed system, some comp onen ts, suc h as a net work link or an individual node, may fail while others still function normally .  This is diÆeren t from the local case, where failures at the system level are alw ays total. Programmers thus have two options: they can either ignore the
% possibilit y of partial failure, resulting in eac h partial failure being unhandled and catastrophic, or they must enhance all of their interfaces to rep ort partial failures adequately , and mak e all of their code prepared for these events. This, however, would mean that local computing becomes more like distributed computing, and not the other way round. ≤ Concurr ency. A similar argumen t can be made for concurrency (parallelism). Unlik e local objects, Waldo et al. say, distributed objects must alw ays be prepared for truly parallel invocations. In a distributed system, there is an actual indeterminacy in the order of metho d invocations, while in the local case, the programmer has complete con trol over invocation order when desired. Additionally , sync hronization becomes much more diffcult in a distributed system, because there is no single point of resource allo cation or sync hronization. Under a unified model, the burden to handle this complexit y would have to be placed on all objects, not just on those where it is actually required.

    A communications protocol is a system of digital message formats and rules for exchanging those messages in or between computing systems and in telecommunications

More informally, a protocol is an agreement between two computer systems on how they will talk to each other.



					% Archivo chap1.tex
%!TEX root = memoria.tex
\chapter{Capítulo 2}

\section{Marco Teórico}

\colorbox{green}{MANTENERSE OBJETIVO EN ESTE CAPÍTULO} \\
\colorbox{pink}{INCLUIR NÚMERO DE RFC EN CADA PROTOCOLO}

Lo que se va a analizar
\begin{itemize}
  \item Protocolos
  \item Serialización
  \item Comunicación entre procesos
\end{itemize}

% ################################################################################################### PROTOCOLOS
\section{Protocolos de Transferencia de Datos}

Para que un dato sea transportado desde un punto a otro necesita tener una forma determinada para ser apto en el mecanismo de transporte, además debe tener una estructura replicable por ambas partes de tal manera que el dato recibido sea una copia fiel del original y sea entendido por ambas partes.

Cuando se trata de la transferencia de datos en arquitecturas de microservicios se debe considerar que el fundamento de todos estos mecanismos es TCP, que pone a disposición el concepto de puertos, para utilizar más de un protocolo por dirección IP; garantiza también el orden e integridad de los grupos de bytes en tránsito, por lo cual, los protocolos pueden construir instrumentos más sofisticados usando estos cimientos.

Un Protocolo de Transferencia de Datos se compone tanto de comandos (esto es un conjunto de reglas que deben ser seguidas en un orden determinado) que son utilizadas en iniciar, negociar, mantener y terminar la conexión entre los puntos involucrados; se compone además de una estructura bien definida y replicable para los datos que han de ser transferidos. Dichas estructuras suelen incluir metadatos, como delimitadores de inicio y términio de un dato o indicadores de la longitud de éstos mismos. Las presencia, ausencia y combinaciones de estos indicadores y su emplazamiento, dan origen a innumerables estructuras posibles (formas de serialización).

Las formas de serialización son elementos intercambiables (y reemplazables) en los protocolos y muy raramente son parte de los protocolos mismos. Esto permite una evolución continua de las arquitecturas y la experimentación con nuevos elementos para optimizar el funcionamiento de los sistemas. De no ser así, esta memoria no sería posible.

\colorbox{green}{MONO: PROTOCOLO = COMANDOS + SERIALIZACION}

Configuraciones comunes en los protocolos de transferencia de datos incluyen métodos de compresión, indicadores configurables de fin de transferencia, indicación del formato de serialización, semánticas de orden y distribución, mecanismos de sincronización, entre otros. Otros parámetros menos usados junto a TCP son las rutinas de detección y corrección de errores.

Es necesario hacer la distinción entre los protocolos de transporte y los protocolos de transmisión de datos. Los primeros son aquellos que determinan la manera en que viajan \textbf{bytes} a través de las redes (contemplan a TCP y UDP); los segundos determinan las maneras en que son transportadas, embaladas y abiertas las \textbf{estructuras de datos} formadas por bytes. Estos últimos son usados por las aplicaciones en la intercomunicación. Un término usado en lengua inglesa es \textit{wire protocol}, cuya traducción literal puede ser protocolo de cable.

Los protocolos de transferencia de datos pueden ser estructurados como comandos de texto o como una secuencia de bytes (también llamado modo binario). La elección de un protocolo basado en texto o uno binario es una decisión arquitectural fundamental, que engloba implicancias en el rendimiento del sistema, complejidad para los desarrolladores de aplicaciones e incluso implicancias a nivel ético y de recursos humanos.

Será la última década en el campo de la informática el foco generacional de esta memoria, de donde se analizan los protocolos relevantes para el desarrollo de software de ésta y la próxima oleada de tecnología. El campo de observación será la evolución temporal que se produce cuando un proyecto nace del revisitar ideas anteriores, como el caso de HTTP/1.1 y HTTP2.

\subsection{Protocolos Basados en Texto}

Cuando los comandos que componen el guión que intercambiarán los sistemas están escritos en lenguajes humanos, estamos en la presencia de protocolos basados en texto, esto es una serie de instrucciones legibles por personas y que deben ser traducidos a formas interpretables por máquinas antes de ser interpretadas por el hardware \colorbox{green}{OJO ACÁ}. Estos protocolos tienen la característica de que no necesitan herramientas de software específicas y su funcionamiento puede ser analizado con técnicas forenses.

La incidencia de los protocolos de transferencia de datos basados en texto es enorme en la actualidad (tan dramática incluso, que se podría metaforizar que son el combustible de lo que hoy conocemos por Internet).

De esta categoría de protocolos, serán estudiados tres de ellos:

\begin{itemize}
  \item STOMP
  \item HTTP
  \item ZMTP
\end{itemize}

Esta elección responde a la necesidad de evaluarles en distintos modelos de comunicación (cliente-servidor, suscripción, punto-a-punto, etc) con el fin de que las mediciones a obtener sean útiles incluso en modelos nuevos.

\subsubsection{STOMP}


STOMP (\textit{Simple/Streaming Text Oriented Messaging Protocol}, Protocolo Sencillo de Mensajes Orientado a Texto). Sus mensajes se componen de un encabezado con propiedades y un cuerpo. Los principios del diseño del protocolo se basan en generar mensajes sencillos y ampliamente inter-operables.

STOMP no maneja colas ni temas; usa una semántica de envío (SEND) con un texto de destino. El intermediario (broker) debe transformar este texto de forma interna en un tema, cola o intercambio. Los consumidores pueden suscribirse a esos destinos. Dado que los destinos no están definidos en la especificación, queda a merced de la implementación el tipo de destino. No siempre es posible cambiar de implementación de manera directa.

Sin embargo, STOMP es sencillo y liviano, con un gran número de librerías en diversos lenguajes. Posee también semánticas transaccionales. Esfuerzos actuales han permitido utilizar el protocolo a través de WebSockets lo que abre la posibilidad de llegar a los navegadores web, dispositivos móviles y aplicaciones en tiempo real.

\subsubsection{HTTP 1.1}
HTTP (\textit{Hypertext Transfer Protocol},  Protocolo de Transferencia de Hipertexto) fue diseñado para sistemas de medios distribuidos y colaborativos. Ha sido usado desde 1990 en la iniciativa de la red amplia mundial (World-Wide Web). La primera versión, conocida como HTTP/0.9, era un protocolo simple para la transferencia de datos en bruto a través de internet. Su iteración posterior, HTTP/1.0 mejoró el protocolo al permitir que los mensajes fuesen codificados usando MIME-types; añadiendo metadatos sobre la información transferida y añadiendo modificaciones en la semántica de petición-respuesta. Sin embargo, HTTP/1.0 poseía falencias; no estaba diseñado para permitir intermediarios (proxies), caché, conexiones persistentes o hosts virtuales.

Los mensajes tienen la siguiente estructura:

\begin{itemize}
  \item \textbf{Línea inicial} Termina con retorno de carro y un salto de línea. Su estructura depende de si se trata de una petición o una respuesta
    \begin{itemize}
      \item Para las peticiones: la acción requerida por el servidor (método de petición) seguido de la URL del recurso y la versión HTTP que soporta el cliente
      \item Para respuestas: La versión del HTTP usado seguido del código de respuesta (que indica que ha pasado con la petición seguido de la URL del recurso) y de la frase asociada a dicho retorno.
    \end{itemize}
  \item \textbf{Cabeceras} Son metadatos y terminan con una línea en blanco. Estas cabeceras le dan gran flexibilidad al protocolo
  \item \textbf{Cuerpo} Es opcional. Su presencia depende de la línea anterior del mensaje y del tipo de recurso al que hace referencia la URL. Típicamente tiene los datos que se intercambian cliente y servidor Por ejemplo para una petición podría contener ciertos datos que se quieren enviar al servidor para que los procese. Para una respuesta podría incluir los datos que el cliente ha solicitado.
\end{itemize}

Ejemplo
\\

\textbf{Petición}
\begin{verbatim}
GET /index.html HTTP/1.1
Host: www.example.com
User-Agent: nombre-cliente
Referer: www.google.com
User-Agent: Mozilla/5.0
Connection: keep-alive
[Línea en blanco]
\end{verbatim}

\textbf{Respuesta}
\begin{verbatim}
HTTP/1.1 200 OK
Date: Fri, 31 Dec 2003 23:59:59 GMT
Content-Type: text/html
Content-Length: 1221

<html>
</html>
\end{verbatim}

\subsubsection{ZMTP}
ZMTP (ZeroMQ Message Transport Protocol) es un protocolo para la capa de transporte para intercambiar mensajes entre dos pares conectados a una capa de transporte para intercambiar mensajes. Se trata de un protocolo de par a par.

Una conexión típica involucra
\begin{itemize}
  \item Ambos pares acuerdan la versión y los mecanismos de seguridad de la conexión al enviarse mensajes y eligiendo continuar la conexión o bien cerrrándola
  \item Se acuerda el mecanismo de seguridad intercambiando cero o más comandos. Si el acuerdo es exitoso, los pares continúan la discusión, de lo contrario uno o ambos cierran la conexión.
  \item Cada par envía al otro metadatos sobre la conexión a modo de último comando. Los pares combrueban los metadatos y deciden si seguir o cerrar la conexión.
  \item Cada par queda listo para intercambiar mensajes con el otro. Cualquiera de ellos puede cerrar la conexión en cualquier momento.
\end{itemize}

\subsection{Protocolos Binarios}

\colorbox{green}{DEFINICION}

Un protocolo binario es aquel diseñado para ser leído por máquinas o software en lugar de personas. Estos protocolos tienen la característica de ser sucintos, lo que se traduce en una potencial ventaja en rendimiento tanto en transmisión como interpretación,

Dichos protocolos requieren herramientas específicas para ayudar en el desarrollo y la detección de problemas. A diferencia de los protocolos basados en texto, estas no pueden ser usadas para más de un protocolo.

\url{http://martin.kleppmann.com/2012/12/05/schema-evolution-in-avro-protocol-buffers-thrift.html}

\begin{itemize}
  \item AMQP
  \item MQTT
  \item Thrift
  \item HTTP2
\end{itemize}

\subsubsection{AMQP}
AMQP (\textit{Advanced Message Queuing Protocol}, Protocolo Avanzado de Mensajería basado en Colas) fue diseñado como un reemplazo abierto frente a soluciones propietarias de mensajería. Dos de las razones más potentes para la adopción de AMQP han sido confiabilidad e interoperabilidad.

El protocolo incluye una gran cantidad de características de mensajería, incluyendo colas persistentes, mensajería de publicación-suscripción basada en tópicos, ruteo flexible, transacciones y seguridad. Los mensajes pueden ser entregados de forma directa, por tema e incluso basándose en las cabeceras.

Existe una gran flexibilidad para ajustar el funcionamiento del protocolo; es posible controlar el acceso a las colas de mensajes y manejar su profundidad, entre otras cosas.

Todo el protocolo se diseñó desde el comienzo para permitir la interoperabilidad entre implementadores.

\textbf{BNF}

\begin{verbatim}
amqp                = protocol-header *amqp-unit 
protocol-header     = literal-AMQP protocol-id protocol-version 
literal-AMQP        = %d65.77.81.80             ; "AMQP" 
protocol-id         = %d0                       ; Must be 0 
protocol-version    = %d0.9.1                   ; 0-9-1 
method              = method-frame [ content ] 
method-frame        = %d1 frame-properties method-payload frame-end 
frame-properties    = channel payload-size 
channel             = short-uint                ; Non-zero 
payload-size        = long-uint 
method-payload      = class-id method-id *amqp-field 
class-id            = %x00.01-%xFF.FF 
method-id           = %x00.01-%xFF.FF 
amqp-field          = BIT / OCTET 
                    / short-uint / long-uint / long-long-uint 
                    / short-string / long-string 
                    / timestamp 
                    / field-table 
short-uint          = 2*OCTET                   
long-uint           = 4*OCTET                   
long-long-uint      = 8*OCTET                   
short-string        = OCTET *string-char        ; length + content 
string-char         = %x01 .. %xFF 
long-string         = long-uint *OCTET          ; length + content 
timestamp           = long-long-uint            ; 64-bit POSIX 
field-table         = long-uint *field-value-pair 
field-value-pair    = field-name field-value 
field-name          = short-string
field-value         = 't' boolean
                    / 'b' short-short-int
                    / 'B' short-short-uint
                    / 'U' short-int
                    / 'u' short-uint
                    / 'I' long-int
                    / 'i' long-uint
                    / 'L' long-long-int
                    / 'l' long-long-uint
                    / 'f' float
                    / 'd' double
                    / 'D' decimal-value
                    / 's' short-string
                    / 'S' long-string
                    / 'A' field-array
                    / 'T' timestamp
                    / 'F' field-table
                    / 'V'                       ; no field 
boolean             = OCTET                     ; 0 = FALSE, else TRUE 
short-short-int     = OCTET 
short-short-uint    = OCTET 
short-int           = 2*OCTET                   
long-int            = 4*OCTET                   
long-long-int       = 8*OCTET                   
float               = 4*OCTET                   ; IEEE-754 
double              = 8*OCTET                   ; rfc1832 XDR double
decimal-value       = scale long-uint
scale               = OCTET                     ; number of decimal digits
field-array         = long-int *field-value     ; array of values
frame-end           = %xCE 
content             = %d2 content-header *content-body 
content-header      = frame-properties header-payload frame-end 
header-payload      = content-class content-weight content-body-size 
                      property-flags property-list 
content-class       = OCTET 
content-weight      = %x00 
content-body-size   = long-long-uint 
property-flags      = 15*BIT %b0 / 15*BIT %b1 property-flags 
property-list       = *amqp-field 
content-body        = %d3 frame-properties body-payload frame-end 
body-payload        = *OCTET 
heartbeat           = %d8 %d0 %d0 frame-end                    
\end{verbatim}

\subsubsection{MQTT}
MQTT (\textit{Message Queue Telemetry Transport}, Cola de Mensajes para Transporte Telerimétrico), fue desarrollado originalmente por IBM junto a otros actores de la industria y liberado años más tarde a la comunidad de código abierto. Actualmente está siendo desarrollado por el consorcio OASIS.

Los principios de diseño y metas de MQTT son mucho más simples que AMQP: entregar un modelo de publicación-suscripción sin colas (a pesar del nombre), donde el énfasis está en el intercambio de información entre dispositivos de recursos limitados, altas latencias y bajos anchos de banda.

\begin{itemize}
  \item Sólo cinco métodos
  \item No es posible agregar metadatos a los mensajes
  \item Encabezados comprimidos
\end{itemize}

Soporta QoS 1, 2 y 3

Facebook usa el protocolo para sus servicios de mensajería ("Facebook Messenger") con gran éxito REFERENCIA

\subsubsection{HTTP2}

HTTP/2 es desarrollado por la división HTTP del Grupo de Trabajo de Ingeniería de Internet (IETF por sus sigla en inglés), quien se encarga de mantener también el protocolo HTTP. Se trata de un esfuerzo conjunto de implementadores, usuarios, operadores de red y expertos en el campo.

A diferencia de su predecesor (HTTP1.1), esta versión ya no está basada en texto plano, si no que se estructura de forma binaria. Dejar las órdenes basadas en texto, sumado a la nueva compresión de encabezados, permite que el protocolo haya reducido dramáticamente la transferencia de datos a cambio de dejar de ser legible por el ojo humano. Sin embargo, ya existen herramientas para facilitar la tarea de analizar el tráfico.

Otra diferencia dramática de esta versión del protocolo es la posibilidad de un cliente de enviar múltiples peticiones usando una misma conexión, es más, el servidor puede responder las peticiones en cualquier orden, lo que permite mantener un flujo contante de respuestas. Es más, el servidor es ahora capaz de enviar recursos al cliente sin que hayan sido explícitamente requeridos. Estos aspectos son muy importantes en escuelas de programación reactiva y arquitecturas de microservicios.

% \begin{description}
% \item [Protocolo Binario]{Binary protocols are more efficient to parse, more compact “on the wire”, and most importantly, they are much less error-prone, compared to textual protocols like HTTP/1.x, because they often have a number of affordances to “help” with things like whitespace handling, capitalization, line endings, blank lines and so on \footnote{IEFT Working Group https://http2.github.io/faq/}.}
% \end{description}

\colorbox{green}{ESPECIFICACIÓN} <-- principalmente ventajas

\colorbox{green}{MÁQUINAS DE ESTADO} <-- mono

\colorbox{green}{QoS} <-- 1,2,3
\colorbox{green}{QUIÉN LO USA} <-- farándula

\subsubsection{Apache Thrift}

Thrift es un proyecto que engloba un protocolo binario orientado a flujos, un lenguaje de definición de interfaces, un conjunto de mecanismos de serialización y herramientas para la generación de código. A diferencia del resto de los proyectos estudiados acá, Thrift intenta entregar una solución completa \colorbox{green}{ARGUMENTAR ACÁ} en la transferencia de mensajes.

Fue desarrollado inicialmente en Facebook para el desarrollo escalable de servicios que abarcan múltiples lenguajes de programación. Más tarde, fue donado a la fundación Apache quien es el actual gobernante del proyecto.

% ############################################################## SERIALIZACIÓN
\section{Serialización}

\subsection{Serialización Basada en Texto}

\colorbox{green}{DEFINIR}

Una forma de serialización común es a través del uso de carácteres de texto como delimitadores de las estructuras de datos o bien, el uso de estructuras de datos usando únicamente texto legible por el humano.

\begin{itemize}
  \item XML
  \item JSON
\end{itemize}

\subsubsection{XML}
XML (Extensible Markup Language, Lenguaje de Marcado Extensible) es un lenguaje basado en etiquetas que define un conjunto de reglas para la codificación de documentos en un formato que a la vez legible por humanos y máquinas. La especificación es gobernada por el Consorcio Mundial de la Web (W3C, World Wide Web Consortium).

Los principios de diseño de XML enfatizan la simplicidad, generalización y usabilidad a través de todo Internet. El formato se basa netamente en texto, con soporte Unicode para permitir el uso amplio de lenguajes humanos. El lenguaje es ampliamente usado para transmitir estructuras de datos a través de servicios webs.

Dado que se trata de un formato extensible, es necesario el uso de esquemas que especifiquen la forma de las estructuras de datos.

\subsubsection{JSON}

La Notación de Objeto de JavaScript (JSON) es un formato ligero para el intercambio de datos. Es sencillo de leer y escribir por humanos; es simple de procesar y generar para las máquinas.

Tiene sus raíces en un subconjunto del Estándar ECMA de JavaScript \colorbox{green}{ECMA-262 3rd Edition - December 1999}. A nivel de programación, es independiente del lenguaje, pero utiliza convenciones familiares para programadores de C, C++, C\#, Java, JavaScript, Perl, Python, entre otros.

Se construye en base a dos estructuras

\begin{itemize}
  \item Una colección de pares clave-valor, conocido en otros lenguajes como un objeto, registro, estructura, diccionario, tablas, listas indexadas or arreglos asociativos
  \item Una lista ordenada de valores: arreglos, vectores, listas o secuencias.
\end{itemize}

\subsection{Serialización Binaria}
\colorbox{green}{DEFINIR}

Acá se hace uso de semánticas al nivel de byte para definir los atributos que permiten generar y procesar las estructuras de datos.

\begin{itemize}
  \item Protocol Buffers
  \item Avro
  \item Thrift
\end{itemize}

\subsubsection{Protocol Buffers}
\subsubsection{Avro}
\subsubsection{Thrift}

\begin{itemize}
  \item BinaryProtocol 
  \item CompactProtocol
\end{itemize}

\section{Patrones de Transferencia de Datos Entre Procesos}

% http://www.tablesgenerator.com/
\begin{table}[]
  \begin{tabular}{@{}lll@{}}
  \toprule
            & Uno a uno            & Uno a muchos                 \\ \midrule
  Síncrono  & Petición-respuesta   &                              \\ \midrule
  Asíncrono & Notificación         & Petición-respuesta asíncrona \\ 
            & Productor-consumidor & Productor asíncrono          \\ \bottomrule
  \end{tabular}
\end{table}

\subsection{Transferencia Síncrona}

\subsubsection{Petición-Respuesta}

\subsection{Transferencia Asíncrona}

\subsubsection{Uno a uno}
\textbf{Notificación}
\textbf{Productor-Consumidor}

\subsubsection{Uno a muchos}
\textbf{Petición/Respuesta asíncrona (push)}
\textbf{Productor asíncrono (suscripción)}

\section{Lenguajes de programación} % (fold)
\label{sec:lenguajes_de_programación}

Por último, incompleto estaría este trabajo al no considerar las implicancias de los paradigmas de programación en la implementación de sistemas y sus métricas a observar.

La principal fuerza modelando la industria en este momento es la programación desde, hacia y mediante la web. Donde se incluye
\begin{itemize}
  \item lenguajes que son diseñados para crear aplicaciones de plataforma mediante el uso de herramientas nativas de la web (orientadas al navegador)
  \item lenguajes que transportan y transforman datos para ser consumidos por aplicaciones web (orientadas a cómputo)
\end{itemize}

\subsubsection{}

% section lenguajes_de_programación (end)					% Archivo chap2.tex
%!TEX root = memoria.tex
\chapter{Capítulo 3}

\section{Diseño y Construcción de las Pruebas}

Para evaluar el rendimiento, complejidad y escalabilidad de cada combinación de protocolo de transferencia, formato de serialización y patrón de comunicación entre procesos, se diseñarán dos arquitecturas de software; una enfocada en operaciones transaccionales y otra con menos garantías.

\section{Metodología}

Mido escalabilidad con peticiones bidireccionales.

Establecer un criterio para diferenciar la manera en que se distribuyen las peticiones dentro del sistema; argumentar que la naturaleza de la arquitectura tiene grandes repercusiuones en la utilidad de los resultados de esta memoria; aún así, es posible generar mediciones que sirvan de aproximación.

Usar una matriz de 3x3 (o R3) con un eje que indique densidad de la arquitectura, el otro indica peticiones y el último indica tipo de distribución (aleatoria, euclidiana, constante)

\colorbox{green}{MODELOS DE DISTRIBUCIÓN}

El uso de modelos de distribución permitir comprobar la solidez de los resultados cuando cambia la arquitectura o demostrar que no son corroborables. Es también una medida para ir controlando los experimentos.

Tomar tiempo hasta el primer byte útil, latencias aportadas por el protocolo usando el mismo broker, mismos servidores, mismos servicios en una misma arquitectura; sólo medir lo que aporta el protocolo y las serializaciones cuando el resto se mantiene constante.

\begin{itemize}
  \item Aleatorio
  \item Euclidiano
  \item Constante (acá hay que crear una forma de generar un patrón aleatorio que se mantenga durante toda la medición)
\end{itemize}

\begin{itemize}
  \item Levantar máquinas
  \item instalar software
  \item escoger apps
  \item programar gateway
  \item programar scheduler de tareas
  \item alguna forma de variar los parámetros
\end{itemize}

\subsection{Procedimiento}
Una vez implementadas las arquitecturas, \colorbox{green}{brokers, }

Apps capaces de intercambiar en tiempo de ejecución los brokers y formatos de serialización.
Implica que las apps deben tener soporte para todo lo estudiado.

\subsection{Instrumentos}

\url{http://www.dell.com/us/business/p/poweredge-r920/pd}
\begin{itemize}
  \item CPU
  \item RAM
  \item Red
  \item Disco
\end{itemize}

\subsubsection{Software}

Para realizar las mediciones se utilizará

\begin{itemize}
  \item Typhoon
  \item Apache jMeter
\end{itemize}

\subsection{Estrategia de análisis}

\subsection{}

\section{Métricas} % (fold)
\label{sec:métricas}

% section métricas (end)
\textbf{Rendimiento}

\begin{itemize}
  \item Mensajes por segundo
  \item Operaciones por segundo
  \item Overhead de transporte
  \item Overhead de serialización
  \item Latencia hasta el primer byte del cuerpo
  \item Latencia hasta el primer byte del cuerpo
\end{itemize}

\textbf{Operacionales}
\begin{itemize}
  \item Infraestructura
  \item Complejidad percibida
  \item Líneas de código
  \item Tareas automatizables
  \item Tareas automatizables
\end{itemize}

\textbf{Documentación}
\begin{itemize}
  \item Estándares
  \item Librerías disponibles
  \item Lenguajes soportados
\end{itemize}					% Archivo chap3.tex
%!TEX root = memoria.tex
\chapter{Capítulo 4}

\section{Mediciones}

INCLUIR ACÁ LOS RESULTADOS DE LAS MEDICIONES
\\
INCLUIR ANÁLISIS INICIAL DE LOS RESULTADOS					% Archivo chap4.tex
%!TEX root = memoria.tex
\chapter{Capítulo 5}

\section{Validación y Análisis de los Resultados}

\textit{\textbf{Recomendaciones generales:}}
\begin{itemize}
  \item \textit{Presenta los resultados interpretando los datos a la luz del marco teórico planteado, dependiendo de la pregunta de investigación y los objetivos de la misma.}
  \item \textit{Recuerda que los datos no son los resultados. Los resultados vienen del análisis de los datos para encontrarles sentido. Se apoyan en los datos, pero son interpretaciones del estudiante que requieren de argumentación para que tengan sentido. Además, se debe presentar la respuesta que el estudiante da a la pregunta de investigación, con la argumentación que lo lleva a dar esta respuesta.}
\end{itemize}

\section{Conclusiones}

\textit{La función principal de las conclusiones es exponer los hallazgos, observaciones y posibles retos que se desprendan de la investigación realizada. Esta sección nunca puede, como parte formal, ser más extensa que la investigación misma. Una buena sección de conclusiones expondrá de manera puntual y ordenada una síntesis de la investigación realizada a partir de los resultados obtenidos en relación con la hipótesis que dio inicio al trabajo. Al exponer los resultados, debe poder apreciarse si se trata de resultados esperados, si difieren de la expectativas, si son parciales o insuficientes. Es decir, los resultados deben referenciarse con el punto de partida de la investigación. Esto hace que, entre la sección de introducción y la de las conclusiones, haya un vínculo evidente.}\\

\textit{Es importante considerar que las conclusiones no constituyen un resumen de los capítulos o partes de la tésis. En esta sección tampoco es apropiado introducir nuevos elementos teóricos ni hay lugar para retomar el objeto de estudio desde otras perspectivas.}\\

\textit{La parte más valiosa de las conclusiones es la reflexión que se deriva de los resultados obtenidos. No es suficiente listar los resultados. A partir de ellos hay qye elaborar observaciones y, sobre todo, nuevas preguntas para futuras investigaciones que permitan aportar en la contrucción del conocimiento.}					% Archivo chap5.tex
%...                  % Agregar aquí más capítulos

%%%%%%%%%%%%%%%%%%%%%%%%%%%%%%%%%%%%%
%	Bibliografía
%%%%%%%%%%%%%%%%%%%%%%%%%%%%%%%%%%%%%
\begin{spacing}{1}
\bibliographystyle{usm}
\bibliography{memoria} % Use 'memoria.bib' | Alternative 'apalike'
\end{spacing}


%%%%%%%%%%%%%%%%%%%%%%%%%%%%%%%%%%%%%
%	Anexos (Opcional)
%%%%%%%%%%%%%%%%%%%%%%%%%%%%%%%%%%%%%
\appendix
%!TEX root = memoria.tex
\chapter{ANEXO 1}\label{appx:licencia}

%...                  % Agregar más apéndices


\end{document}
