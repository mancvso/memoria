%!TEX root = memoria.tex

% introducir arquitecturas de microservicios, promesas y problemas
Las Arquitecturas de Microservicios proponen la separación de grandes sistemas en pequeños componentes de responsabilidad reducida que poseen su propio almacén de datos, una interfaz de comunicación bien definida y son desarrollados por equipos pequeños e independientes. Un sistema de mediana embergadura suele poseer decenas de microservicios, los que requieren un entorno computacional adaptable a la carga del sistema y estrategias de comunicación entre ellos.
% presentar conclusiones arquitectónicas
Cada mensaje que viaja de un microservicio a otro debe ser serializado en el origen, transportado a través de la red, ruteado y deserializado en destino. La elección de tecnologías y estrategias para realizar dichas operaciones es vital para el perfecto desempeño del sistema y de los programadores, cuyos criterios dependen de la naturaleza del problema. Sin embargo, la estabilización del mercado de herramientas permite generar una matriz que ayude a los arquitectos de software a escoger las herramientas que han probado ser aptas en la industria y alejarse de aquellas combinaciones que han sido insuficientes o sobre complejizadas.
% presentar conclusiones de desarrollo de sistemas
La correcta elección puede acelerar el desarrollo de software, optimizar recursos humanos y mejorar la comunicación entre equipos, además de permitir la introducción de nuevos programadores de manera controlada, aislada y fomentar la innovación dentro de las empresas.

\paragraph{Palabras Claves:} 
Arquitecturas de Microservicios, serialización binaria, protocolos binarios, desarrollo de software basado en mensajes

\vspace{20mm}
